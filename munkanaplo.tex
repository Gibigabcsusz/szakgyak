\documentclass[10pt,a4paper,oneside]{report}
% nagyon sok kép esetén meggyorsítható a fordítás a draft móddal
% \documentclass[12pt,a4paper,oneside,draft]{report}
% ekkor a képek nem renderelődnek ki, csak placeholder lesz mérethelyesen
\usepackage[utf8]{inputenc} % mindenképp maradjon az utf-8 kódolás
\usepackage[magyar]{babel}
\usepackage[T1]{fontenc}
%\usepackage{amsmath}
%\usepackage{amsfonts}
%\usepackage{amssymb}
%\usepackage{graphics} % grafikus elemek, képek berakásához
%\usepackage{epsfig} % eps importáláshoz
%\usepackage{listings}
%\usepackage{sectsty}
%\usepackage{enumerate}
%\usepackage{siunitx} % ezzel lehet hivatalosan megformázni: szám + mértékegység
%\usepackage{lastpage}
\usepackage{setspace}
\usepackage{hyperref} % PDF hivatkozásokhoz kell
%\usepackage[hang]{caption}
%\usepackage{titling} % a title, author parancsok szabad használatához
\usepackage{xcolor}
\usepackage{multicol}
\usepackage{blindtext}
\usepackage{wrapfig}


\pagenumbering{gobble}


\definecolor{rosewood}{rgb}{0.6, 0.0, 0.04}
\definecolor{indigo(dye)}{rgb}{0.0, 0.25, 0.42}

% az A4 oldal margóinak és méreteinek beállítása
\usepackage[left=15mm,right=15mm,top=10mm,bottom=5mm]{geometry}\pagestyle{plain}

% A sorköz távolság beállítása
% egyszeres sorköz
\singlespacing
% 1,5 sorköz
% \onehalfspacing

% A hivatkozások, és linkek átállítása alapértelmezett színre, fekete-fehér nyomtatáshoz optimalizálva
\hypersetup
{
  	colorlinks,
  	citecolor=blue,
 	linkcolor=rosewood,
  	urlcolor=indigo(dye)
}

\setlength{\columnsep}{1 cm}

\newcounter{magicrownumbers}
\newcommand\rownum{\stepcounter{magicrownumbers}\arabic{magicrownumbers}}


\begin{document}

\begin{center}
	\Large{Szakmai Gyakorlat Munkanapló}
\end{center}
\begin{tabular}{p{2.5 cm} p{2.5 cm} p{5 cm} p{6 cm}}
	\textbf{Név:} & Szilágyi Gábor & \textbf{Gyakorlat Időtartama:} & X hét\\
	\textbf{NEPTUN:} & NOMK01 & \textbf{Intézmény:} & Silicon Laboratories Hungary Kft.\\
	\textbf{Képzés:} & MSc & \textbf{Intézmény székhelye:} & 1033 Budapest, Ángel Sanz Briz út 13. \\
	& & \textbf{Gyakorlóhelyi Konzulens:} & Bódi Tamás
\end{tabular}
\begin{table}[h!]
	\centering
	\small
	\begin{tabular}{| c | p{3 cm} | p{10 cm} |}
	\hline
	 & Dátum & Tevékenység (Megjegyzés: Az egyes sorok 4 órát jelentenek) \\ \hline \hline
	\rownum & 2022.01.17. & Mérési hiba okának keresése \\ \hline
	$\sum$ & Y óra& \\ \hline
	\end{tabular}
\end{table}
\vspace{.5 cm} \\
Gyakorlóhelyi konzulens:

\end{document}